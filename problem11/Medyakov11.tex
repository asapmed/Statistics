\documentclass[14pt]{extarticle}

\usepackage[T2A]{fontenc}
\usepackage[utf8]{inputenc}
\usepackage[russian]{babel}
\usepackage{graphicx}
\usepackage{caption}
\DeclareCaptionLabelSeparator{dot}{. }
\captionsetup{justification=centering,labelsep=dot}
\usepackage{amsmath}
\usepackage{amssymb}
\input glyphtounicode.tex
\input glyphtounicode-cmr.tex
\pdfgentounicode=1
\usepackage{bm}

\begin{document}

ФИО: Медяков Даниил Олегович

\vspace{10pt}

Номер задачи: 11

\vspace{10pt}

Решение:

\vspace{10pt}

Введем следующие случайные величины.

$\xi_1$ - случайная величина, принимающая три значения: заниженный размер детали из \textbf{первой} партии, точный размер детали из \textbf{первой} партии, завышенный размер детали из \textbf{первой} партии.

$\xi_2$ - случайная величина, принимающая три значения: заниженный размер детали из \textbf{второй} партии, точный размер детали из \textbf{второй} партии, завышенный размер детали из \textbf{второй} партии.

Наша цель проверить гипотезу о том, что размер детали не зависит от партии, и это эквивалентно гипотезе $H_1$: величины $\xi_1$ и $\xi_2$ имеют одинаковые распределения. Для этого используем критирей однородности $\chi^2$. Он применим, так как объемы выборки для каждой из партий больше 50 и каждая из частот больше 5. Дополнительно положим уровень значимости $\alpha = 0.05$, так как он не задан по условию задачи.

Для $\xi_1$ и $\xi_2$ нам даны следующие выборки с $N = 3$:

\begin{eqnarray}
\notag
\begin{cases}
    \nu_1 = \{25, 50, 25\}, \overline{p}_1 = \{p_{11}, p_{12}, p_{13}\}\\
    \nu_2 = \{52, 41, 7\}, \overline{p}_2 = \{p_{21}, p_{22}, p_{23}\}
\end{cases}
\end{eqnarray}

Далее найдем $\hat{P} = \arg\underset{p_j}{\max}\prod\limits_j (p_j)^{\nu_{\cdot j}} \Leftrightarrow \hat{P}_j = \frac{\nu_{\cdot j}}{n}$, где $n = n_1 + n_2$.

Причем, $\nu_{\cdot 1} = 25 + 52 = 77, \nu_{\cdot 2} = 50 + 41 = 91, \nu_{\cdot 3} = 25 + 7 = 32$.

Тогда, $\hat{P}_1 = 0.385, \hat{P}_2 = 0.455, \hat{P}_3 = 0.16$.

Теперь мы готовы вычислить статистику критерия:

\begin{equation}
\notag
    T_{\chi^2} = \sum\limits_{i=1}^k\sum\limits_{j=1}^N\frac{\left(\nu_{ij} - n_i \hat{P}_j\right)^2}{n_i \hat{P}_j} = 10.24 + 10.24 = 20.48.
\end{equation}


При том, что наша гипотеза $H_1$ верна, мы получаем сходимость $T_{\chi^2}\overset{H_1, d}{\underset{n_1, n_2\rightarrow\infty}{\longrightarrow}} \chi^2((N-1)(k-1))$. В тоже время $t_{\alpha}$ удовлетворяет условию на уровень значимости, то есть $\mathbb P(T_{\chi^2} \geq t_{\alpha}) = \alpha\Rightarrow \mathbb P(T_{\chi^2} < t_\alpha) = F_{\chi^2} (t_{\alpha}) = 1 - \alpha$. Тогда $t_{\alpha}$ суть $(1-\alpha)$ квантиль распределения $\chi^2((N-1)(k-1))$. В нашем случае:

\begin{equation*}
    t_{0.05} = \chi^2(2)_{0.95} = 5.99.
\end{equation*}

Получили $T_{\chi^2} = 20.48 > 5.99 = t_{\alpha}$, то есть мы попадаем в критическую область $\Omega_{\text{кр.}} = \{x\in\Omega ~|~ T_{\chi^2}(x) \geq t_{\alpha}\}$, а значит гипотезу $H_1$ отклоняем.

Таким образом, гипотезу о независимости номера партии деталей
и размера детали мы отклоняем на уровне значимости $\alpha = 0.05$.

\end{document}