\documentclass[14pt]{extarticle}

\usepackage[T2A]{fontenc}
\usepackage[utf8]{inputenc}
\usepackage[russian]{babel}
\usepackage{graphicx}
\usepackage{caption}
\DeclareCaptionLabelSeparator{dot}{. }
\captionsetup{justification=centering,labelsep=dot}
\usepackage{amsmath}
\usepackage{amssymb}
\input glyphtounicode.tex
\input glyphtounicode-cmr.tex
\pdfgentounicode=1
\usepackage{bm}

\begin{document}

ФИО: Медяков Даниил Олегович

\vspace{10pt}

Номер задачи: 72

\vspace{10pt}

Решение:

\vspace{10pt}

По гипотезе $H_0$ все элементы выборки равномерно распределены на отрезке $[0, 2]$, по гипотезе $H_1$ все элементы выборки равномерно распределены на отрезке $[1, 3]$. Нахождение такого критерия $\pi$, что $\max(\alpha, \beta)$ минимально эквивалентно нахождению минимаксного решающего правила, где матрица штрафов имеет вид 

\begin{equation*}
    c = 
    \begin{pmatrix}
      0& 1\\
      1& 0
    \end{pmatrix},
\end{equation*}

тогда $R_1(\pi) = \alpha, R_2(\pi) = \beta$.

По лемме Неймана-Пирсона критерий

\begin{equation*}
    \pi_{c,p} = 
    \begin{cases}
        1, l(x) > c\\
        p, l(x) = c\\
        0, l(x) < c
    \end{cases},
\end{equation*}

при $c > 0, p\in[0, 1], R_1(\pi_{c,p}) = R_2(\pi_{c,p})$ определяет минимаксное решающее правило. 

Функция правдоподобия:

\begin{equation*}
    l(x) = \frac{\mathbb I(x\in[1, 3]^n)}{\mathbb I(x\in[0, 2]^n)} = 
    \begin{cases}
        1, x\in[1, 2]^n\\
        0, \exists x_i\in[0,1]; x\in[0,2]^n\\
        +\infty, \exists x_i\in[2,3]; x\in[1,3]^n\\
        \text{не определена, иначе}
    \end{cases}
\end{equation*}

Возьмем $c = 1$. Тогда

$\alpha = \mathbb P_1(l(x) > c) + p\mathbb P_1(l(x) = c) = p\mathbb P_1(\forall i\hookrightarrow x_i\in[1, 2]) = \frac{p}{2^n}$,

$\beta = \mathbb P_2(l(x) < c) + (1 - p)\mathbb P_2(l(x) = c) = (1 - p)\mathbb P_2(\forall i\hookrightarrow x_i\in[1, 2]) = \frac{1 - p}{2^n}$.

Из требований к лемме Неймана-Пирсона получаем $\frac{p}{2^n} = \frac{1 - p}{2^n}\Rightarrow p = \frac{1}{2}$.

Тогда искомый критерий:

\begin{equation*}
    \pi = 
    \begin{cases}
        1, l(x) > 1\\
        \frac{1}{2}, l(x) = 1\\
        0, l(x) < 1
    \end{cases}.
\end{equation*}

\end{document}