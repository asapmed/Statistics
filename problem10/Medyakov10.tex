\documentclass[14pt]{extarticle}

\usepackage[T2A]{fontenc}
\usepackage[utf8]{inputenc}
\usepackage[russian]{babel}
\usepackage{graphicx}
\usepackage{caption}
\DeclareCaptionLabelSeparator{dot}{. }
\captionsetup{justification=centering,labelsep=dot}
\usepackage{amsmath}
\usepackage{amssymb}
\input glyphtounicode.tex
\input glyphtounicode-cmr.tex
\pdfgentounicode=1
\usepackage{bm}

\begin{document}

ФИО: Медяков Даниил Олегович

\vspace{10pt}

Номер задачи: 10

\vspace{10pt}

Решение:

\vspace{10pt}

Для начала проверим, можем ли мы пользоваться критерием согласия $\chi^2$ для сложной гипотезы для этой задачи. Поскольку $n = 200 \geq 50$ и $\nu = \{10, 181, 9\}$, то есть каждая $\nu_j \geq 5$, то мы можем пользоваться критерием согласия $\chi^2$ для сложной гипотезы. Заметим, что по условию нам не задан уровень значимости, а значит возьмем $\alpha = 0,05$. Мы проверяем гипотезу $H_1: Bi(2, \theta)$, а значит, исходя из этой гипотезы и условия, мы имеем следующие параметры:

\begin{eqnarray}
\notag
\begin{cases}
    N = 3\\
    p_0^0 = C_2^0 \theta^0 (1 - \theta)^2\\
    p_1^0 = C_2^1 \theta (1 - \theta)\\
    p_2^0 = C_2^2 \theta^2 (1 - \theta)^0\\
    n = 200\\
    \nu = \{10, 181, 9\}
\end{cases}
\end{eqnarray}

Далее нам необходимо найти $\hat{\theta}\in \mathbb R^d: \hat{\theta} = \arg\underset{\theta}{\max}\left[\prod\limits_{j = 0}^{N-1} p_j^0 (\theta)^{\nu_j}\right] \Leftrightarrow \sum\limits_{j = 0}^{N-1} \frac{\nu_j}{p_j^0(\theta)}\cdot\frac{\partial p_j^0 (\theta)}{\partial \theta_k}, k = \overline{1, r}$. В нашей задаче $r = 1$. Тогда получаем:

\begin{eqnarray}
\notag
    \begin{split}
        -\frac{\nu_0}{(1-\theta)^2}2(1-\theta) + \frac{\nu_1}{2\theta(1-\theta)}2(1 - 2\theta) + \frac{\nu_2}{\theta^2}2\theta = 0
        \\
        -2\nu_0\theta + \nu_1(1 - 2\theta) + 2\nu_2(1-\theta) = 0\Rightarrow \hat{\theta} = \frac{\nu_1 + 2\nu_2}{2n} = 0.4975
    \end{split}
\end{eqnarray}


Теперь мы готовы вычислить статистику $T_{\chi^2}$:

\begin{equation}
\notag
    T_{\chi^2} = \sum\limits_{j=1}^N\frac{\left(\nu_j - np_j^0(\hat{\theta})\right)^2}{np_j^0(\hat{\theta)}} = 32,48 + 65,62 + 33,14 = 131,24
\end{equation}


При том, что наша гипотеза $H_1$ верна, мы получаем сходимость $T_{\chi^2}\overset{H_1, d}{\underset{n\rightarrow\infty}{\longrightarrow}} \chi^2(N-1-r)$. В тоже время $t_{\alpha}$ удовлетворяет условию на уровень значимости, то есть $\mathbb P(T_{\chi^2} \geq t_{\alpha}) = \alpha\Rightarrow \mathbb P(T_{\chi^2} < t_\alpha) = F_{\chi^2} (t_{\alpha}) = 1 - \alpha$. Тогда $t_{\alpha}$ суть $(1-\alpha)$ квантиль распределения $\chi^2(N-1-r)$. В нашем случае:

\begin{equation*}
    t_{0,05} = \chi^2(1)_{0,95} = 3,84.
\end{equation*}

Получили $T_{\chi^2} = 131,24 > 3,84 = t_{\alpha}$, то есть мы попадаем в критическую область $\Omega_{\text{кр.}} = \{x\in\Omega ~|~ T_{\chi^2}(x) \geq t_{\alpha}\}$, а значит гипотезу $H_1$ отклоняем.

\end{document}