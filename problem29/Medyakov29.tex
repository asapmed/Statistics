\documentclass[14pt]{extarticle}

\usepackage[T2A]{fontenc}
\usepackage[utf8]{inputenc}
\usepackage[russian]{babel}
\usepackage{graphicx}
\usepackage{caption}
\DeclareCaptionLabelSeparator{dot}{. }
\captionsetup{justification=centering,labelsep=dot}
\usepackage{amsmath}
\usepackage{amssymb}
\input glyphtounicode.tex
\input glyphtounicode-cmr.tex
\pdfgentounicode=1
\usepackage{bm}

\begin{document}

ФИО: Медяков Даниил Олегович

\vspace{10pt}

Номер задачи: 29

\vspace{10pt}

Решение:

\vspace{10pt}

Рассматриваем $\mathcal{U}[\theta, 2\theta], \theta > 0$. Будем искать достаточную статистику малой размерности. Для начала запишем функцию правдоподобия:

\begin{equation*}
    L(\mathbf{X}, \theta) = \frac{1}{\theta}\cdot\mathbb I (X_1\in [\theta, 2\theta])\cdot\ldots\cdot\frac{1}{\theta}\cdot\mathbb I (X_n\in[\theta, 2\theta]) = \frac{1}{\theta^n}\cdot\mathbb I(\mathbf{X}\in[\theta, 2\theta]^n).
\end{equation*}

Тот факт, что $\mathbf{X}\in[\theta, 2\theta]^n$ эквивалентно тому, что максимальный из $X_i$ меньше $2\theta$, а минимальный из $X_i$ больше $\theta$. Тогда это условие перепишем через порядковые статистики:

\begin{equation*}
\begin{cases}
    X_{(1)} \geqslant \theta\\
    X_{(n)} \leqslant 2\theta
\end{cases}
\end{equation*}

Тогда функция правдоподобия принимает вид:

\begin{equation*}
    L(\mathbf{X}, \theta) = \frac{1}{\theta^n}\cdot \mathbb I(X_{(1)} \geqslant \theta)\cdot \mathbb I(X_{(n)} \leqslant 2\theta)
\end{equation*}

Положим $g(S(\mathbf{X}), \theta) = L(\mathbf{X}, \theta)$. Тогда по критерию факторизации статистика

\begin{equation*}
    S(\mathbf{X}) = \left(X_{(1)}, X_{(n)}\right)
\end{equation*}

является достаточной с размерностью 2.

\end{document}