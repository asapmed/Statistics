\documentclass[14pt]{extarticle}

\usepackage[T2A]{fontenc}
\usepackage[utf8]{inputenc}
\usepackage[russian]{babel}
\usepackage{graphicx}
\usepackage{caption}
\DeclareCaptionLabelSeparator{dot}{. }
\captionsetup{justification=centering,labelsep=dot}
\usepackage{amsmath}
\usepackage{amssymb}
\input glyphtounicode.tex
\input glyphtounicode-cmr.tex
\pdfgentounicode=1
\usepackage{bm}

\begin{document}

ФИО: Медяков Даниил Олегович

\vspace{10pt}

Номер задачи: 22

\vspace{10pt}

Решение:

\vspace{10pt}

$X\in \mathcal{U}[a, a+1]$ - случайная величина, $X_1, \ldots X_n$ - выборка размера $n$.

1. $A_1^* = \overline{X} - \frac{1}{2}$. 

Для начала проверим эту статистику на несмещенность:

\begin{equation*}
    \mathbb E_a \left[\overline{X} - \frac{1}{2}\right] = \mathbb E_a \left[\overline{X}\right] - \frac{1}{2} = \frac{1}{n}n\left(a + \frac{1}{2}\right) - \frac{1}{2} = a
\end{equation*}

Получили, что статистика обладает свойством несмещенности. Теперь проверим на состоятельность. Для этого необходимо

\begin{equation*}
    \forall a~~ T_n(\textbf{X})\overset{\mathbb P_a}{\underset{n\rightarrow+\infty}{\longrightarrow}}a \Leftrightarrow \forall a~~\forall\varepsilon~~ \underset{n\rightarrow\infty}{\lim} \mathbb P_a\left[|T_n(\textbf{X}) - a| > \varepsilon\right] = 0
\end{equation*}

Для оценки слагаемого под пределом будем пользоваться неравенством Чебышева, то есть

\begin{equation*}
    \forall \varepsilon > 0 ~~\mathbb P\left[|\xi - \mathbb E\xi| > \varepsilon\right] \leqslant \frac{\mathbb D\xi}{\varepsilon^2},
\end{equation*}

где $\xi$ - случайная величина с конечным математическим ожиданием и дисперсией. В нашем случае, для $\mathcal{U}[a, a+1]$ мы получаем:

\begin{gather*}
    \mathbb D_a[A_1^*] = \frac{1}{n^2}\mathbb D_a\left[\sum\limits_{i=1}^n x_i\right] = \frac{1}{n^2}\sum\limits_{i=1}^n \mathbb D_a[x_i] = \frac{n}{n^2}\frac{1}{12} = \frac{1}{12n} < +\infty,\\
    \forall \varepsilon > 0 ~~ \mathbb P_a[|A_1^* - a| > \varepsilon] \leqslant \frac{\mathbb D[A_1^*]}{\varepsilon^2} = \frac{1}{12\varepsilon^2n}\underset{n\rightarrow\infty}{\longrightarrow}0.
\end{gather*}

Получили сходимость по вероятности к нулю, а значит $A_1^*$ состоятельна.

2. $A_2^* = X_{(n)} - \frac{n}{n+1}$. $X_{(n)} - a\in Beta(n, 1)$, где для $\xi\in Beta(\alpha, \beta)$ верно:

\begin{equation*}
    \mathbb E \xi = \frac{\alpha}{\alpha + \beta},\quad \mathbb D\xi = \frac{\alpha\beta}{(\alpha + \beta)^2(\alpha + \beta + 1)}.
\end{equation*}

Проверим на несмещенность:

\begin{gather*}
    \mathbb E_a[A_2^*] = \mathbb E_a[X_{(n)}] - \frac{n}{n+1} = \mathbb E_{Beta(n, 1)}[X_{(n)} - a] + a - \frac{n}{n+1} = \frac{n}{n+1} +\\
    + a - \frac{n}{n+1} = a.
\end{gather*}

Свойство несмещенности выполняется.

Аналогично первому пункту проверим состоятельность:

\begin{equation*}
    \mathbb D_a[A_2^*] = \mathbb D_a[X_{(n)} - a] = \mathbb D_{Beta(n, 1)}[\xi] = \frac{n}{(n+1)^2(n+2)}\underset{n\rightarrow\infty}{\longrightarrow}0.
\end{equation*}

Это значит, что можно ограничить выражение под пределом для сходимости по вероятности с помощью неравенства Чебышева и сам этот предел равен нулю, то есть сходимость по вероятности есть. Значит свойство состоятельности выполняется.

Теперь, для того, чтобы выяснить, какая статистика является наиболее предпочтительной, сравним их дисперсии:

\begin{equation*}
    \frac{1}{12n}~~ \textbf{VS}~~ \frac{n}{(n+1)^2(n+2)} \Leftrightarrow n^3 - 8n^2 + 5n + 2 ~~\textbf{VS}~~ 0.
\end{equation*}

Получили, что при $1\leqslant n \leqslant 8$ лучше использовать первую статистику, иначе вторую.

\end{document}