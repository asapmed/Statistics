\documentclass[14pt]{extarticle}

\usepackage[T2A]{fontenc}
\usepackage[utf8]{inputenc}
\usepackage[russian]{babel}
\usepackage{graphicx}
\usepackage{caption}
\DeclareCaptionLabelSeparator{dot}{. }
\captionsetup{justification=centering,labelsep=dot}
\usepackage{amsmath}
\usepackage{amssymb}
\input glyphtounicode.tex
\input glyphtounicode-cmr.tex
\pdfgentounicode=1
\usepackage{bm}

\begin{document}

ФИО: Медяков Даниил Олегович

\vspace{10pt}

Номер задачи: 17

\vspace{10pt}

Решение:

\vspace{10pt}

Будем пользоваться методом моментов. Будем выражать оцениваемый параметр через моменты

\begin{equation*}
    \hat{\alpha}_m = \frac{x_1^m + \ldots + x_n^m}{n}
\end{equation*}

Для нормального распределения, имеем:

\begin{equation*}
    \begin{cases}
        \hat{\alpha}_1 = \mathbb E\xi = m\\
        \hat{\alpha}_2 = \mathbb D\xi = \sigma^2\\
    \end{cases}.
\end{equation*}

Тогда искомое $m$ следующее:

\begin{equation*}
    m = \hat{\alpha}_1 = \frac{1}{n}\sum\limits_{i=1}^n x_i = \frac{1}{n}\sum\limits_{i=1}^n \ln v_i = \overline{\ln V}.
\end{equation*}

\end{document}