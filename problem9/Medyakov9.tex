\documentclass[14pt]{extarticle}

\usepackage[T2A]{fontenc}
\usepackage[utf8]{inputenc}
\usepackage[russian]{babel}
\usepackage{graphicx}
\usepackage{caption}
\DeclareCaptionLabelSeparator{dot}{. }
\captionsetup{justification=centering,labelsep=dot}
\usepackage{amsmath}
\usepackage{amssymb}
\input glyphtounicode.tex
\input glyphtounicode-cmr.tex
\pdfgentounicode=1
\usepackage{bm}

\begin{document}

ФИО: Медяков Даниил Олегович

\vspace{10pt}

Номер задачи: 9

\vspace{10pt}

Решение:

\vspace{10pt}

Для начала проверим, можем ли мы пользоваться критерием согласия $\chi^2$ для этой задачи. Поскольку $n = 800 \geq 50$ и $\nu = \{74, 92, 83,\\ 79, 80, 73, 77, 75, 76, 91\}$, то есть каждая $\nu_j \geq 5$, то мы можем пользоваться критерием согласия $\chi^2$. Заметим, что по условию нам не задан уровень значимости, а значит возьмем $\alpha = 0,05$. Мы проверяем гипотезу $H_1$ о равномерном распределении данных, а значит, исходя из этой гипотезы и условия, мы имеем следующие параметры:

\begin{eqnarray}
\notag
\begin{cases}
    N = 10\\
    p_j^0 = \frac{1}{10}, j=\overline{1, 10}\\
    n = 800\\
    \nu = \{74, 92, 83, 79, 80, 73, 77, 75, 76, 91\}
\end{cases}
\end{eqnarray}

Теперь мы готовы вычислить статистику $T_{\chi^2}$:

\begin{eqnarray}
\notag
\begin{split}
    T_{\chi^2} = \sum\limits_{j=1}^N\frac{\left(\nu_j - np_j^0\right)^2}{np_j^0} =
    \\= 
    \frac{6^2 + 12^2 + 3^2 + 1^2 + 0^2 + 7^2 + 3^2 + 5^2 + 4^2 + 11^1}{80} = 5,125.
\end{split}
\end{eqnarray}


При том, что наша гипотеза $H_1$ верна, мы получаем сходимость $T_{\chi^2}\overset{H_1, d}{\underset{n\rightarrow\infty}{\longrightarrow}} \chi^2(N-1)$. В тоже время $t_{\alpha}$ удовлетворяет условию на уровень значимости, то есть $\mathbb P(T_{\chi^2} \geq t_{\alpha}) = \alpha\Rightarrow \mathbb P(T_{\chi^2} < t_\alpha) = F_{\chi^2} (t_{\alpha}) = 1 - \alpha$. Тогда $t_{\alpha}$ суть $(1-\alpha)$ квантиль распределения $\chi^2(N-1)$. В нашем случае:

\begin{equation*}
    t_{0,05} = \chi^2(9)_{0,95} = 16,9.
\end{equation*}

Получили $T_{\chi^2} = 5,125 < 16,9 = t_{\alpha}$, то есть мы не попадаем в критическую область $\Omega_{\text{кр.}} = \{x\in\Omega ~|~ T_{\chi^2}(x) \geq t_{\alpha}\}$, а значит гипотезу $H_1$ не отклоняем.

\end{document}