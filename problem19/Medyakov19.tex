\documentclass[14pt]{extarticle}

\usepackage[T2A]{fontenc}
\usepackage[utf8]{inputenc}
\usepackage[russian]{babel}
\usepackage{graphicx}
\usepackage{caption}
\DeclareCaptionLabelSeparator{dot}{. }
\captionsetup{justification=centering,labelsep=dot}
\usepackage{amsmath}
\usepackage{amssymb}
\input glyphtounicode.tex
\input glyphtounicode-cmr.tex
\pdfgentounicode=1
\usepackage{bm}

\begin{document}

ФИО: Медяков Даниил Олегович

\vspace{10pt}

Номер задачи: 19

\vspace{10pt}

Решение:

\vspace{10pt}

$l$ - истинная длина стержня, тогда, с учетом ошибки, измеренная длина имеет распределение $\mathcal{N}(l, kl)$. Тогда функция правдоподобия

\begin{equation*}
    L(x, l) = \left(\frac{1}{\sqrt{2\pi kl}}\right)^n\cdot \exp \left(-\frac{1}{2kl}\sum\limits_{i=1}^n (x_i - l)^2\right)
\end{equation*}

Максимизация $L$ эквивалентна максимизации $\ln L$ из-за монотонности логарифма и $L \geqslant 0$. 

\begin{equation*}
    \ln L(x, l) = -\frac{n}{2}\cdot\ln(2\pi kl)-\frac{1}{2kl}\sum\limits_{i=1}^n (x_i - l)^2
\end{equation*}

\begin{gather*}
    \frac{\partial \ln L}{\partial l} = -\frac{n}{2}\frac{2\pi k}{2\pi kl} - (-1)\frac{1}{2kl^2}\sum\limits_{i=1}^n (x_i - l)^2 - \frac{1}{2kl}(-2)\sum\limits_{i=1}^n (x_i - l) = 0\\
    -\frac{n}{2l}+\frac{1}{2kl^2}\sum\limits_{i=1}^n (x_i - l)^2 + \frac{1}{kl}\sum\limits_{i=1}^n (x_i - l) = 0\\
    -nkl + \sum\limits_{i=1}^n (x_i - l)^2 + 2l\sum\limits_{i=1}^n (x_i - l) = 0\\
    -nkl + \sum\limits_{i=1}^n (x_i^2 - 2x_i l - l^2) + \sum\limits_{i=1}^n (2lx_i - 2l^2) = 0\\
    -nkl + \sum\limits_{i=1}^n (x_i^2 - l^2) = 0; -nkl - nl^2 + \sum\limits_{i=1}^n x_i^2 = 0; l = \frac{-k \pm \sqrt{k^2 + 4\overline{x^2}}}{2}
\end{gather*}

Поскольку $l$ - длина стержня, то нас интересует только положительный корень. Покажем, что эта точка является глобальным максимумом. Для этого посмотрим на знак второй производной.

\begin{gather*}
    \frac{\partial^2 \ln L}{\partial^2 l} = \frac{\partial}{\partial l}\left(\frac{-nkl - nl^2 + \sum\limits_{i=1}^n x_i^2}{2kl^2}\right) = \frac{-nk - 2nl}{2kl^2} +\\
    + \frac{-nkl - 2 \sum\limits_{i=1}^n x_i^2 - nl^2}{2kl^3} = \frac{1}{2kl^3}\left(-nkl - 2nl^2 + 2nkl - 2\sum\limits_{i=1}^n x_i^2 + 2nl^2\right) =\\
    =\frac{1}{2kl^3}\left(nkl - 2\sum\limits_{i=1}^n x_i^2\right)
\end{gather*}

Выпуклости на $l\in(0, +\infty)$ будет достаточно для того, чтобы найденная точка являлась максимумом из-за гладкости $\ln L$. То есть,

\begin{equation*}
    \frac{\partial^2 \ln L}{\partial^2 l} < 0 \Leftrightarrow \frac{kl}{2} < \frac{1}{n}\sum\limits_{i=1}^n x_i^2 = \overline{x^2}.
\end{equation*}

Получили, что при выполнении условия $kl < 2\overline{x^2}$ ОМП:

\begin{equation*}
    l = \frac{-k + \sqrt{k^2 + 4\overline{x^2}}}{2}.
\end{equation*}

\end{document}