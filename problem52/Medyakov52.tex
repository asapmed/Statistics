\documentclass[14pt]{extarticle}

\usepackage[T2A]{fontenc}
\usepackage[utf8]{inputenc}
\usepackage[russian]{babel}
\usepackage{graphicx}
\usepackage{caption}
\DeclareCaptionLabelSeparator{dot}{. }
\captionsetup{justification=centering,labelsep=dot}
\usepackage{amsmath}
\usepackage{amssymb}
\input glyphtounicode.tex
\input glyphtounicode-cmr.tex
\pdfgentounicode=1
\usepackage{bm}

\begin{document}

ФИО: Медяков Даниил Олегович

\vspace{10pt}

Номер задачи: 52

\vspace{10pt}

Решение:

\vspace{10pt}

По условию:

\begin{gather*}
    H_1: \mathcal{N} (-1, 1)\\
    H_2: Exp(2)
\end{gather*}

Для начала запишем риски:

\begin{gather*}
    R_1 = -1\cdot\mathbb P_1(H_1) + 2\cdot\mathbb P_1(H_2) = 2\alpha - (1-\alpha) = 3\alpha - 1\\
    R_2 = -1\cdot\mathbb P_2(H_2) + 2\cdot\mathbb P_2(H_1) = 2\beta - (1-\beta) = 3\beta - 1
\end{gather*}

Далее запишем функцию отношения правдоподобия:

\begin{gather*}
    l(x_1, x_2) = \left[4\exp(-2(x_1 + x_2))\mathbb I (x_1, x_2 > 0)\right]\cdot\\
    \cdot\left[\frac{1}{2\pi}\exp(-\frac{1}{2}((x_1 + 1)^2 + (x_2 + 1)^2))\right]^{-1}
\end{gather*}

В случае если $x_i\leqslant 0$ хотя бы для одного из $i$, то $l(x_1, x_2) = 0$.

В противном случае:

$l(x_1, x_2) = 8\pi\exp\left[\frac{1}{2}((x_1 + 1)^2 + (x_2 + 1)^2 - 4(x_1 + x_2))\right]$

Таким образом в функция отношения правдоподобия записывается в следующем виде:

\begin{equation*}
    l(x_1, x_2) = 
    \begin{cases}
        8\pi\exp\left[\frac{1}{2}((x_1 - 1)^2 + (x_2 - 1)^2)\right], x_1, x_2 > 0\\
        0, \text{иначе}
    \end{cases}
\end{equation*}

Запишем критерий:

\begin{equation*}
    \pi = 
    \begin{cases}
        1, l(x_1, x_2) \geqslant c\\
        0, l(x_1, x_2) < c
    \end{cases}
\end{equation*}

Согласно лемме Неймана-Пирсона в полной постановке этот критерий является байесовским при $c = \frac{c_{21} - c_{11}}{c_{12} - c_{22}}\frac{q_1}{q_2} = \frac{3}{2}$.

При этом байесовский риск

\begin{equation*}
    r(\pi) = R_1(\pi)q_1 + R_2(\pi)q_2 = \frac{3}{5}(3\alpha - 1) + \frac{2}{5}(3\beta - 1)
\end{equation*}

Теперь найдем ошибку первого рода:

\begin{gather*}
    \alpha = \mathbb P_1\left[l(x)\geqslant\frac{3}{2}\right] = \mathbb P_1\left[x_1, x_2 > 0, (x_1 - 1)^2 + (x_2 - 1)^2 \geqslant \underbrace{2\ln(\frac{3}{16\pi})}_{ < 0}\right] =\\
    =\mathbb P_1 [x_1 > 0, x_2 > 0] = (1 - F_{\mathcal{N}(-1, 1)}^2 (0))^2 = 0,025.
\end{gather*}

Теперь найдем ошибку второго рода: 

\begin{gather*}
    \beta = \mathbb P_2\left[l(x) < \frac{3}{2}\right] = \mathbb P_2\left[(x_1 - 1)^2 + (x_2 - 1)^2 < 0\right]+\\
    + \mathbb P_2\left[x_1 < 0 ~\text{или}~ x_2 < 0\right] = 0.
\end{gather*}

\end{document}