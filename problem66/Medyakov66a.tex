\documentclass[14pt]{extarticle}

\usepackage[T2A]{fontenc}
\usepackage[utf8]{inputenc}
\usepackage[russian]{babel}
\usepackage{graphicx}
\usepackage{caption}
\DeclareCaptionLabelSeparator{dot}{. }
\captionsetup{justification=centering,labelsep=dot}
\usepackage{amsmath}
\usepackage{amssymb}
\input glyphtounicode.tex
\input glyphtounicode-cmr.tex
\pdfgentounicode=1
\usepackage{bm}

\begin{document}

ФИО: Медяков Даниил Олегович

\vspace{10pt}

Номер задачи: 66a

\vspace{10pt}

Решение:

\vspace{10pt}

Не умаляя общности, положим $c_{12} = c_{21} = 1$.

Поскольку матрица штрафов имеет вид 
\begin{equation*}
    c = 
    \begin{pmatrix}
      0& 1\\
      1& 0
    \end{pmatrix},
\end{equation*}

то риски равны ошибкам первого и второго рода, так как

\begin{gather*}
    R_1(\delta) = c_{11}p_{11} + c_{21}p_{21} = \alpha\\
    R_2(\delta) = c_{12}p_{12} + c_{22}p_{22} = \beta
\end{gather*}

Выпишем область, в которой байесовское решающее правило принимает $H_2$:

\begin{gather*}
    h_1(x) \geqslant \sum\limits_{j=1}^2 c_{1j}q_1f_1(x)\\
    l(x) \geqslant \frac{c_{21} - c_{11}}{c_{12} - c_{22}}\frac{q_1}{q_2} = 1
\end{gather*}

Тогда решающее правило:

\begin{gather*}
    \delta(x)=
    \begin{cases}
        1, l(x) \geqslant 1\\
        0, l(x) < 1
    \end{cases}
\end{gather*}

Рассматриваем первую компоненту. В этом случае случайная величина имеет одно из двух нормальных распределений: $H_{1a}: \mathcal{N}(1, 1), H_{2a}: \mathcal{N}(-1, 1)$.

$l(x) = \exp(\frac{1}{2}((x-1)^2 - (x+1)^2)) = \exp(-2x)$.

Тогда $\alpha = \mathbb P_1(l(x)\geqslant 1) = \mathbb P_1(\exp(-2x)\geqslant 1) = \mathbb P_1(x\leqslant 0) = F_{\mathcal{N}(1, 1)}(0) = 0,1587$.

$\beta = \mathbb P_2(l(x) < 1) = \mathbb P_2(x > 0) = 1 - F_{\mathcal{N}(-1, 1)}(0) = 0,1587$

Тогда минимальный риск:

$r(\delta) = R_1(\delta)q_1 + R_2(\delta)q_2 = \frac{\alpha + \beta}{2} = 0,1587$.

\end{document}