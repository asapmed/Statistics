\documentclass[14pt]{extarticle}

\usepackage[T2A]{fontenc}
\usepackage[utf8]{inputenc}
\usepackage[russian]{babel}
\usepackage{graphicx}
\usepackage{caption}
\DeclareCaptionLabelSeparator{dot}{. }
\captionsetup{justification=centering,labelsep=dot}
\usepackage{amsmath}
\usepackage{amssymb}
\input glyphtounicode.tex
\input glyphtounicode-cmr.tex
\pdfgentounicode=1
\usepackage{bm}

\begin{document}

ФИО: Медяков Даниил Олегович

\vspace{10pt}

Номер задачи: 66c

\vspace{10pt}

Решение:

\vspace{10pt}

Не умаляя общности, положим $c_{12} = c_{21} = 1$.

Поскольку матрица штрафов имеет вид 
\begin{equation*}
    c = 
    \begin{pmatrix}
      0& 1\\
      1& 0
    \end{pmatrix},
\end{equation*}

то риски равны ошибкам первого и второго рода, так как

\begin{gather*}
    R_1(\delta) = c_{11}p_{11} + c_{21}p_{21} = \alpha\\
    R_2(\delta) = c_{12}p_{12} + c_{22}p_{22} = \beta
\end{gather*}

Выпишем область, в которой байесовское решающее правило принимает $H_2$:

\begin{gather*}
    h_1(x) \geqslant \sum\limits_{j=1}^2 c_{1j}q_1f_1(x)\\
    l(x) \geqslant \frac{c_{21} - c_{11}}{c_{12} - c_{22}}\frac{q_1}{q_2} = 1
\end{gather*}

Тогда решающее правило:

\begin{equation*}
    \delta(x)=
    \begin{cases}
        1, l(x) \geqslant 1\\
        0, l(x) < 1
    \end{cases}
\end{equation*}

\begin{gather*}
    l(x)=\frac{\exp\left(-\frac{1}{2} \cdot(x-m_2)^T R^{-1}(x-m_2)\right)}{\exp\left(-\frac{1}{2} \cdot(x-m_1)^T R^{-1}(x-m_1)\right)} =
    \\
    = \exp (\frac{1}{3} \cdot(2x_1^2-2x_1-x_1x_2-2x_1+2+x_2-x_1x_2+x_2+2x_2^2 - \\ - 2x_1^2-2x_1+x_1x_2-2x_1-2+x_2+x_1x_2+x_2-2x_2^2))
    = \\
    =\exp \left(\frac{1}{3} \cdot\left(-8x_1+4x_2 \right)\right) = \exp \left(\left(-\frac{8}{3}x_1+\frac{4}{3}x_2 \right)\right)
\end{gather*}

Запишем ошибки первого и второго рода:

\begin{gather*}
\alpha=\mathbb{P}_1\left[l\left(x\right) \geqslant 1\right]=\mathbb{P}_1\left[\exp \left(-\frac{8}{3}x_1+\frac{4}{3}x_2 \right) \geqslant 1\right]=\mathbb{P}_1\left[-\frac{8}{3}x_1+\frac{4}{3}x_2 \geqslant 0\right]=\\
=\mathbb{P}_1\left[2x_1 - x_2 \leqslant 0\right] \\
\beta=\mathbb{P}_2[l(x)<1]=\mathbb{P}_2[x_2-2x_1<0]
\end{gather*}
Плотность вероятности нормального распределения может быть записана следующим образом:
$$
f_1(x_1, x_2) = \frac{1}{2\pi\sqrt{0.75}} \exp{\left(-\frac{1}{2}\left(\begin{array}{cc}x_1 - 1 & x_2\\ \end{array}\right)\left(\begin{array}{cc}1 & 0.5\\ 0.5 & 1\\ \end{array}\right)^{-1}\left(\begin{array}{c}x_1 - 1\\ x_2\\ \end{array}\right)\right)}.
$$
$$
f_2(x_1, x_2) = \frac{1}{2\pi\sqrt{0.75}} \exp{\left(-\frac{1}{2}\left(\begin{array}{cc}x_1 + 1 & x_2\\ \end{array}\right)\left(\begin{array}{cc}1 & 0.5\\ 0.5 & 1\\ \end{array}\right)^{-1}\left(\begin{array}{c}x_1 + 1\\ x_2\\ \end{array}\right)\right)}.
$$
Теперь интеграл, который нам нужно вычислить, записывается следующим образом:

$$\int\int_D f(x_1, x_2) d(x_1, x_2),$$

где $D$ - область, где выполняется условие $2x_1 - x_2 \leq 0$. Тогда
\begin{gather*}
\alpha = 0.180397 \\
\beta = 0.0744975
\end{gather*}

Минимальный байесовский риск:
$$
r(\delta)= \frac{0.180397 + 0.0744975}{2} = 0,1274
$$

\end{document}