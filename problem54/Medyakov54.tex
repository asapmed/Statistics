\documentclass[14pt]{extarticle}

\usepackage[T2A]{fontenc}
\usepackage[utf8]{inputenc}
\usepackage[russian]{babel}
\usepackage{graphicx}
\usepackage{caption}
\DeclareCaptionLabelSeparator{dot}{. }
\captionsetup{justification=centering,labelsep=dot}
\usepackage{amsmath}
\usepackage{amssymb}
\input glyphtounicode.tex
\input glyphtounicode-cmr.tex
\pdfgentounicode=1
\usepackage{bm}

\begin{document}

ФИО: Медяков Даниил Олегович

\vspace{10pt}

Номер задачи: 54

\vspace{10pt}

Решение:

\vspace{10pt}

Запишем функцию правдоподобия обеих гипотез:

\begin{gather*}
    L_1(x) = \frac{1}{9}\mathbb{I}(x\in[0,3]^2)\\
    L_2(x) = 0,25\exp(-\frac{1}{2}(x_1 + x_2))\mathbb I(x_1\geqslant 0)\mathbb I(x_2\geqslant 0)
\end{gather*}

Тогда 
\begin{equation*}
    l(x) = \frac{0,25\exp(-\frac{1}{2}(x_1 + x_2))\mathbb I(x_1\geqslant 0)\mathbb I(x_2\geqslant 0)}{\frac{1}{9}\mathbb{I}(x\in[0,3]^2)}
\end{equation*}

Рассмотрим несколько случаев:

\begin{enumerate}
    \item $x\in [0,3]^2$. Тогда $l(x) = \frac{9}{4}\exp(-\frac{1}{2}(x_1 + x_2))$.
    \item $x_i \in [0,3], x_j\notin[0,3]$. Тогда, если $x_j \geqslant 0$, то $l(x) = +\infty$.
    \item $x_i, x_j \notin [0,3]$. Тогда, если $x_i, x_j \geqslant 0$, то $l(x) = +\infty$.
\end{enumerate}

Таким образом,

\begin{equation*}
    l(x) = 
    \begin{cases}
        \frac{9}{4}\exp(-\frac{1}{2}(x_1 + x_2)), x\in[0,3]^2\\
        +\infty, x\in [0, +\infty]^2\backslash[0,3]^2
    \end{cases}
\end{equation*}

\begin{equation*}
    \pi_{c, p}(x) = 
    \begin{cases}
        1, l(x) > c\\
        p, l(x) = c\\
        0, l(x) < c
    \end{cases}
\end{equation*}

\begin{equation*}
    \mathbb P_{H_1}(l(x) > c) + p\mathbb P_{H_1}(l(x) = c) = \alpha
\end{equation*}

При $c<0$, выражение равно 1 и решений уравнения соответствующих критерию Неймана-Пирсона не существует для заданного $\alpha$. 

При $c>0$ :

\begin{gather*}
\mathbb{P}_{H_{1}}(l(x)>c)= \\
\mathbb{P}_{H_{1}}\left(2.25 \exp \left(-\frac{1}{2}\left(x_{1}+x_{2}\right)\right)>c\right) = \\
\mathbb{P}_{H_{1}}\left(l(x)=\frac{9}{4} \exp \left(-\frac{1}{2}\left(x_{1}+x_{2}\right)\right)\right)+P_{H_{1}}(l(x)=+\infty)= \\
=\mathbb{P}_{H_{1}}\left(\frac{9}{4} \exp \left(-\frac{1}{2}\left(x_{1}+x_{2}\right)\right)>c\right)= \\
=\mathbb{P}_{H_{1}}\left(-\left(x_{1}+x_{2}\right)>2 \ln \left(\frac{4}{9} c\right)\right)= \\
=\mathbb{P}_{H_{1}}\left(x_{1}+x_{2}<\tilde{c}\right)
\end{gather*}

Последнее выражение геометрически означает площадь под пересечением прямой и квадрата. Так как нас интересует хоть какое то $\tilde{c}$, то ограничемся случаем, когда получившееся фигура -- треугольник. То есть:

$$
\mathbb{P}_{H_{1}}\left(x_{1}+x_{2}<\tilde{c}\right)=\frac{1}{2} \tilde{c}^{2}
$$

Аналогично:

$$
\mathbb{P}_{H_{1}}(l(x)=c)=\mathbb{P}_{H_{1}}\left(x_{1}+x_{2}=\tilde{c}\right)=0,
$$

причём $\mathbb{P}_{H_{1}}\left(x_{1}+x_{2}=\tilde{c}\right)=0$, в силу того, что $x_{1}+x_{2}$ распределено непрерывно.

Тогда получим:

$$
\frac{1}{2} \tilde{c}^{2}=\alpha \Longrightarrow \tilde{c}=2 \sqrt{\alpha}
$$

$p$ получилось произвольным, поэтому возьмём $p=0$.

Получили, 

\begin{equation*}
    \pi(x) = 
    \begin{cases}
        1, x_{1}+x_{2}<2 \sqrt{\alpha}, \text {или} x_{1}>3, \text { or } x_{2}>3\\
        0, x_{1}+x_{2} \geq 2 \sqrt{\alpha}
    \end{cases}
\end{equation*}

\begin{gather*}
\beta(\pi)=1-\mathbb{E}_{2} \pi(X)=1-\mathbb{P}_{2}\left(x_{1}+x_{2}<2 \sqrt{\alpha}\right)-\mathbb{P}_{2}\left(x_{1}>3\right)-\mathbb{P}_{2}\left(x_{2}>3\right)= \\
=1-\Gamma(2,2)_{2 \sqrt{\alpha}}-\left(1-1+\exp \left(-\frac{3}{2}\right)\right)-\left(1-1+\exp \left(-\frac{3}{2}\right)\right)= \\
=1-0.009-2 \cdot 0.223=0.545
\end{gather*}

\end{document}